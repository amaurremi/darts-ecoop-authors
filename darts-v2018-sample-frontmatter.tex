%This is a template for producing the frontmatter of an issue in DARTS.

\documentclass[a4paper,UKenglish]{dartsmaster}
 %for A4 paper format use option "a4paper", for US-letter use option "letterpaper"
 %for british hyphenation rules use option "UKenglish", for american hyphenation rules use option "USenglish"

\usepackage{microtype}%if unwanted, comment out or use option "draft"
\usepackage{wrapfig}

%\editor{John Q. Open \\ Joan R. Access} %optional to modify editors on first page

\EventEditors{Maria Christakis, Philipp Haller, and Marianna Rapoport}
\EventNoEds{3}
\EventLongTitle{32nd European Conference on Object-Oriented Programming (ECOOP 2018)}
\EventShortTitle{ECOOP 2018}
\EventAcronym{ECOOP}
\EventYear{2018}
\EventDate{July 15--21, 2018}
\EventLocation{Amsterdam, Netherlands}
\EventLogo{}
\SeriesVolume{4}
\SeriesIssue{3}
\ArticleNo{0} % the frontmatter is always the first paper and has always the article number 0 (zero).
\DatePublished{July 2018}


\begin{document}

\frontmatter

%%
%% PAGE 1: Cover page
%%%

\maketitle

%%
%% PAGE 2: Bibliographic data (editors, ACM classification, ISBN, license, DOI, ...)
%% 

\begin{publicationinfo}%for page ii, please fill as required
\sffamily
\twocolumn

{\Large\bf\sffamily \textbf{\href{http://www.dagstuhl.de/lites}{ISSN \printISSN{}}}}

\bigskip

\newcommand{\orcid}[1]{\url{http://orcid.org/#1}}
\newcommand{\email}[1]{\href{mailto:#1}{\texttt{#1}}}

\emph{DARTS Special Issue Editors} \\[0.2cm]
Maria Christakis \\
Max Planck Institute for Software Systems \\
Kaiserslautern, Germany\\ 
\email{maria@mpi-sws.org}\\

Philipp Haller\\
KTH Royal Institute of Technology\\
Stockholm, Sweden\\ 
\email{phaller@kth.se}\\

Marianna Rapoport\\
University of Waterloo\\
Waterloo, Canada\\
\email{mrapoport@uwaterloo.ca}

\bigskip

\emph{Published online and open access by}\newline
Schloss Dagstuhl -- Leibniz-Zentrum f\"ur Informatik GmbH, Dagstuhl Publishing, Saarbr\"ucken/Wadern, Germany. 

Online available at \\ \url{http://drops.dagstuhl.de/darts}.

\bigskip
\emph{Publication date}\newline
\printDatePublished{}



\bigskip

%\emph{Bibliographic information published by the Deutsche Nationalbibliothek}\newline
%The Deutsche Nationalbibliothek lists this publication in the Deutsche Nationalbibliografie; detailed bibliographic data are available in the Internet at \href{http://dnb.d-nb.de}{http://dnb.d-nb.de}. 

\bigskip

\emph{License}\newline
This work is licensed under a Creative Commons Attribution 3.0 Germany license (CC BY~3.0~DE): \href{http://creativecommons.org/licenses/by/3.0/de/deed.en}{\nolinkurl{http://creativecommons.org/licenses/by/}}\linebreak \href{http://creativecommons.org/licenses/by/3.0/de/deed.en}{\nolinkurl{3.0/de/deed.en}}.
\begin{wrapfigure}[2]{l}{1.8cm}
\vspace*{-1\baselineskip}
\includegraphics[width=1.8cm]{cc-by}
\end{wrapfigure} 
In brief, this license authorizes each and everybody to share (to
copy, distribute and transmit) the work under the following
conditions, without impairing or restricting the authors'
moral rights:
\begin{itemize}
\item Attribution: The work must be attributed to its authors.
\end{itemize}

The copyright is retained by the corresponding authors.

%\bigskip
\vfill
\emph{Digital Object Identifier}\newline
\printForewordDOI

\newpage

\vphantom{{\Large\bf\sffamily \textbf{\href{http://www.dagstuhl.de/lites}{ISSN \printISSN{}}}}}~~

\bigskip

\emph{Aims and Scope}\newline
The Dagstuhl Artifacts Series (DARTS) publishes evaluated research data and artifacts in all areas of computer science. An artifact can be any kind of content related to computer science research, e.g., experimental data, source code, virtual machines containing a complete setup, test suites, or tools.

%\medskip

%\bigskip

%\emph{Editorial Board}
%\begin{itemize}
%\item tba
%\end{itemize}
\vfill


\emph{Editorial Office}\newline
Michael Wagner \emph{(Managing Editor)}\\
Jutka Gasiorowski \emph{(Editorial Assistance)}\\
Dagmar Glaser \emph{(Editorial Assistance)}\\
Thomas Schillo \emph{(Technical Assistance)}

\bigskip
\emph{Contact}\newline
Schloss Dagstuhl -- Leibniz-Zentrum f\"ur Informatik\\
DARTS, Editorial Office\\
Oktavie-Allee, 66687 Wadern, Germany\\ 
publishing@dagstuhl.de


\bigskip

\url{http://www.dagstuhl.de/darts}
 
 \thispagestyle{empty}
 \onecolumn

\newpage

\end{publicationinfo}

%%
%% PAGE 5 and more: TOC etc.
%% 

%\begin{dedication}%please fill or comment out
%  Insert dedication here.
%\end{dedication}


\begin{contentslist}
%To generate the table of contents copy all the .vtc files
%of the contributions to your working directory.
%For every contribution type a line
%\inputtocentry{dummycontribution}
%where the argument of \inputtocentry is the name of
%the vtc file without suffix.

%Alternatively write e.g.
\contitem
\title{Preface}
\author{Maria Christakis, Philipp Haller, and Marianna Rapoport}
\page{0:vii}

\contitem
\title{Artifact Evaluation Process}
\author{ }
\page{0:ix}

\contitem
\title{Artifact Evaluation Committee}
\author{ }
\page{0:xi}

\contitem
\title{List of Authors}
\author{ }
\page{0:xiii}

%\part{} %use if volume is divided in parts
\part{Artifacts}

\inputtocentry{p01-campos}
\inputtocentry{p02-toman}
\inputtocentry{p03-oostvogels}
\inputtocentry{p04-inoue}
\inputtocentry{p05-bi}
\inputtocentry{p06-krueger}
\inputtocentry{p07-milanova}
\inputtocentry{p08-mezzetti}
\inputtocentry{p09-bi}
\inputtocentry{p10-chung}


\end{contentslist}

\chapter{Preface} %please fill or comment out

The artifact evaluation (AE) committee reviews software artifacts and data sets that accompany research papers published at ECOOP. The goals of the committee are to ensure that the reviewed artifacts are reproducible, well-documented, and closely correspond to the associated paper. 

The AE process for 2018 closely resembled the work done for ECOOP 2017. The artifact evaluation guidelines by Shriram Krishnamurthi, Matthias
Hauswirth, Steve Blackburn, and Jan Vitek published on the Artifact
Evaluation site (\url{http://www.artifact-eval.org}) were of great help. Additionally, this year we created new guidelines for reviewers and authors of artifacts that contain mechanized proofs (\url{http://proofartifacts.github.io/}). 

This year, the committee evaluated 13 artifacts (which correspond to 50\% of all
accepted papers), and accepted 10 of these (a 77\% acceptance rate). In total, 38\% of the research papers published at ECOOP 2018 will receive an artifact-evaluation badge.

The accepted artifacts will be archived in the Dagstuhl Artifacts Series (DARTS) published on the Dagstuhl Research Online Publication Server (DROPS). Each artifact will be assigned a digital object identifier (DOI) that can be used in future citations. 

We would like to thank the 20 members of this year's committee, who donated their valuable time and effort to make the AE process possible. We would also like to thank  Michael Wagner for the publication of the artifacts volume, and the Program Chair Todd Millstein for helping us coordinate the AE with the paper review.

\chapter{Artifact Evaluation Process}
The authors of all papers that were accepted to ECOOP 2018 had the option to submit an artifact with their paper.
Each artifact was evaluated by three reviewers who were part of the artifact evaluation committee. 
The reviewing process consisted of two phases.
In the ``kick-the-tires'' phase, reviewers briefly verified the basic integrity, documentation, and set-up of the artifacts.
In case of any issues, reviewers had the opportunity to ask clarifying questions to the authors.
Authors, in turn, could respond to the reviewers' first feedback, and provide missing documentation
or small fixes to the artifacts, to ensure that reviewers were able to fully evaluate the artifacts.
In the second phase, each reviewer had three weeks to do a comprehensive evaluation of the three artifacts they were
assigned to review. This included assessing whether an artifact fully corresponded to the paper,
whether all results presented in the paper could be reproduced, how well the artifact was documented,
and how easy it would be to re-use the artifact in future research.

\begin{participants}

\chapter[Committee]{Artifact Evaluation Committee}
%use \participant for every author, eg.:
\participant Maria Christakis \\
Max Planck Institute for Software Systems \\
Kaiserslautern, Germany\\ 
maria@mpi-sws.org\\

\participant Philipp Haller\\
KTH Royal Institute of Technology\\
Stockholm, Sweden\\ 
phaller@kth.se\\

\participant Marianna Rapoport\\
University of Waterloo\\
Waterloo, Canada\\
mrapoport@uwaterloo.ca\\

\participant	Ambrose Bonnaire-Sergeant	\\	Indiana University	\\	Bloomington, IN, USA	\\	abonnairesergeant@gmail.com	\\

\participant	Elias Castegren	\\	Uppsala University	\\	Uppsala, Sweden	\\	elias.castegren@it.uu.se	\\

\participant	Ezgi Çiçek	\\	MPI-SWS	\\	Saarbruecken, Germany	\\	ecicek@mpi-sws.org	\\

\participant	Ankush Desai	\\	UC Berkeley	\\	Berkeley, CA, USA	\\	ankush@eecs.berkeley.edu	\\

\participant	Jon Eyolfson	\\	University of Waterloo	\\	Waterloo, Canada	\\	jonathan.eyolfson@uwaterloo.ca	\\

\participant	Yu Feng	\\	UT Austin	\\	Austin, TX, USA	\\	yufeng@cs.utexas.edu	\\

\participant	Thomas Gilray	\\	University of Maryland	\\	College Park, MD, USA	\\	thomas.gilray@gmail.com	\\

\participant	Stefan Heule	\\	Stanford University	\\	Stanford, CA, USA	\\	sheule@cs.stanford.edu	\\

\participant	Hugo Feree	\\	University of Kent	\\	Kent, UK	\\	H.Feree@kent.ac.uk	\\

\participant	Ravichandhran Madhavan	\\	EPFL	\\	Lausanne, Switzerland	\\	ravi.kandhadai@epfl.ch	\\

\participant	Guillaume Martres	\\	EPFL	\\	Lausanne, Switzerland	\\	guillaume.martres@epfl.ch	\\

\participant	Gianluca Mezzetti	\\	Aarhus University	\\	Aarhus, Denmark	\\	mezzetti@cs.au.dk	\\

\participant	Fabian Muehlboeck	\\	Cornell University	\\	Ithaca, NY, USA	\\	fabianm@cs.cornell.edu	\\

\participant	Filip Niksic	\\	MPI-SWS	\\	Kaiserslautern, Germany	\\	fniksic@mpi-sws.org	\\

\participant	Alceste Scalas	\\	Imperial College London	\\	London, UK	\\	alceste.scalas@imperial.ac.uk	\\

\participant	Emma Tosch	\\	UMass Amherst	\\	Amherst, MA, USA	\\	etosch@cs.umass.edu	\\

\participant	Ming-Ho Yee	\\	Northeastern University	\\	Boston, MA, USA	\\	mh@mhyee.com	\\

\end{participants} 


\end{document}
