\documentclass[a4paper,UKenglish]{darts-v2018}
 
\usepackage{microtype}%if unwanted, comment out or use option "draft"

\nolinenumbers % to disable line numbers

%\bibliographystyle{plainurl}% the recommended bibstyle

% Commands for artifact descriptions
% Written by Camil Demetrescu and Erik Ernst
% April 8, 2014

\newenvironment{scope}{\section{Scope}}{}
\newenvironment{content}{\section{Content}}{}
\newenvironment{getting}{\section{Getting the artifact} The artifact 
endorsed by the Artifact Evaluation Committee is available free of 
charge on the Dagstuhl Research Online Publication Server (DROPS).}{}
\newenvironment{platforms}{\section{Tested platforms}}{}
\newcommand{\license}[1]{{\section{License}#1}}
\newcommand{\mdsum}[1]{{\section{MD5 sum of the artifact}#1}}
\newcommand{\artifactsize}[1]{{\section{Size of the artifact}#1}}

\newcommand{\orcid}[1]{\url{http://orcid.org/#1}}
\newcommand{\email}[1]{\href{mailto:#1}{\texttt{#1}}}

% Author macros::begin %%%%%%%%%%%%%%%%%%%%%%%%%%%%%%%%%%%%%%%%%%%%%%%%
\title{Legato: An At-Most-Once Analysis with Applications to Dynamic Configuration Updates (Artifact)\footnote{This work was partially supported under DARPA agreement number FA8750-16-2-0032.}}
%\titlerunning{A Sample DARTS Research Description} %optional, in case that the title is too long; the running title should fit into the top page column

% ARTIFACT: Authors may not be exactly the same as the related scholarly paper, e.g., you may want to include authors who contributed to the preparation of the artifact, but not to the companion paper
\author{John Toman}{Paul G. Allen School of Computer Science \& Engineering, University of
Washington, USA}{jtoman@cs.washington.edu}{}{}{}
\author{Dan Grossman}{Paul G. Allen School of Computer Science \& Engineering, University of
  Washington, USA}{djg@cs.washington.edu}{}{}{}
\authorrunning{J. Toman and D. Grossman}
\Copyright{John Toman and Dan Grossman}

\subjclass{\ccsdesc[500]{Software and its engineering~Automated static analysis}}

\keywords{Static Analysis, Dynamic Configuration Updates}
% Author macros::end %%%%%%%%%%%%%%%%%%%%%%%%%%%%%%%%%%%%%%%%%%%%%%%%%

% Please provide information to the related scholarly information
\RelatedArticle{John Toman and Dan Grossman, ``Legato: An At-Most-Once Analysis with Applications to Dynamic Configuration Updates'', in Proceedings of the 32nd European Conference on Object-Oriented Programming (ECOOP 2018), LIPIcs, Vol.~109, pp.~24:1--24:31, 2018.\newline \url{http://dx.doi.org/10.4230/LIPIcs.ECOOP.2018.24}}

%Editor-only macros:: begin (do not touch as author)%%%%%%%%%%%%%%%%%%%%%%%%%%%%%%%%%%
\Volume{4}
\Issue{3}
\Article{1}
\RelatedConference{32nd European Conference on Object-Oriented Programming (ECOOP 2018), July 19--21, 2018, Amsterdam, Netherlands}
% Editor-only macros::end %%%%%%%%%%%%%%%%%%%%%%%%%%%%%%%%%%%%%%%%%%%%%%%

\begin{document}

\maketitle

\begin{abstract}
  This artifact supports Legato, an at-most-once analysis. An
  at-most-once analysis ensures that an application never observes
  inconsistent versions of its environment by checking that every
  value depends on at most one access of every external resource used
  by the application. We have applied this general analysis to the
  problem of finding errors in applications that support dynamic
  configuration updates (DCU), i.e., configuration updates that are
  applied immediately without program restart. When configurations may
  change at any point during execution, the enforcing the at-most-once
  condition for each configuration option guarantees that the program
  never observes inconsistent versions of configuration options. This
  artifact recreates our experiments, which applied Legato to 10
  applications that support DCU and found several bugs across 9 of the
  10 programs.
\end{abstract}

% ARTIFACT: section on the scope of the artifact (what claims of the paper are intended to be backed by this artifact?)
\begin{scope}
  This artifact is intended to be used to reproduce the experiments in
  the companion paper.  in particular, this artifact includes the
  infrastructure to support the following claims in the paper:
  \begin{itemize}
  \item Legato finds dynamic resource consistency errors in our
    benchmark suite with a manageable ratio of true to false positives.
  \item Legato has reasonable time and memory requirements.
  \end{itemize}

  In particular, this artifact can reproduce Tables 1 and 2, and
  Figures 12--15. In addition, the artifact can reproduce statistics
  reported in the paper, such as peak memory usage. Claim 1 is
  supported by Table 2 and Figures 12--14, and Claim 2 is supported by
  Figure 15 and the statistics computed by the artifact.
\end{scope}

% ARTIFACT: section on the contents of the artifact (code, data, etc.)
\begin{content}
The artifact package includes:
\begin{itemize}
\item A README PDF describing how to setup to the artifact and
  recreate the experiments in the paper.
\item A user guide (found in \texttt{doc/}) describing how to run
  Legato on other programs besides those in the paper.
\item A virtual machine image (found in \texttt{vm/}) that contains
  all of the infrastructure required to recreate the experiments in the paper.
\end{itemize}
\end{content}

% ARTIFACT: section containing links to sites holding the
% latest version of the code/data, if any
\begin{getting}
The source of Legato is also available at \url{https://github.com/uwplse/legato}.
\end{getting}

% ARTIFACT: section specifying the platforms on which the artifact is known to
% work, including requirements beyond the operating system such as large
% amounts of memory or many processor cores
\begin{platforms}
  The core of the artifact is contained in a virtual machine,
  which is provided in the Open Virtualization Archive format.
  This format should work with any modern virtualization software,
  and has been tested with VirtualBox 5.1.34 running on Ubuntu 16.04.
  The virtual machine requires at least 10GB of physical RAM,
  7GB of hard drive space, and a processor with 2 physical cores for the
  virtual machine.
\end{platforms}

% ARTIFACT: section specifying the license under which the artifact is
% made available
\license{The artifact is available under the MIT License.}

% ARTIFACT: section specifying the md5 sum of the artifact master file
% uploaded to the Dagstuhl Research Online Publication Server, enabling 
% downloaders to check that the file is the expected version and suffered 
% no damage during download.
\mdsum{618a2a99707f1b283b2dd01abe0b68d8}

% ARTIFACT: section specifying the size of the artifact master file uploaded
% to the Dagstuhl Research Online Publication Server
\artifactsize{2.67 GiB}

\end{document}