\documentclass[a4paper,UKenglish]{darts-v2018}
 
\usepackage{microtype}


\bibliographystyle{plainurl}


\newenvironment{scope}{\section{Scope}}{}
\newenvironment{content}{\section{Content}}{}
\newenvironment{getting}{\section{Getting the artifact} The artifact 
endorsed by the Artifact Evaluation Committee is available free of 
charge on the Dagstuhl Research Online Publication Server (DROPS).}{}
\newenvironment{platforms}{\section{Tested platforms}}{}
\newcommand{\license}[1]{{\section{License}#1}}
\newcommand{\mdsum}[1]{{\section{MD5 sum of the artifact}#1}}
\newcommand{\artifactsize}[1]{{\section{Size of the artifact}#1}}

\newcommand{\orcid}[1]{\url{http://orcid.org/#1}}
\newcommand{\email}[1]{\href{mailto:#1}{\texttt{#1}}}

\title{ContextWorkflow: A Monadic DSL for Compensable and Interruptible Executions (Artifact)}


\author{Hiroaki Inoue}{Graduate School of Informatics, Kyoto University, Kyoto, Japan}{hinoue@fos.kuis.kyoto-u.ac.jp}{}{The current affiliation is Mitsubishi Electric Corporation.}
\author{Tomoyuki Aotani}{School of Computing, Tokyo Institute of Technology, Tokyo, Japan}{aotani@c.titech.ac.jp}{}{}
\author{Atsushi Igarashi}{Graduate School of Informatics, Kyoto University, Kyoto, Japan}{igarashi@kuis.kyoto-u.ac.jp}{}{}

\authorrunning{H. Inoue, T. Aotani and A. Igarashi}
\Copyright{Hiroaki Inoue, Tomoyuki Aotani and Atsushi Igarashi}

\subjclass{
\ccsdesc[500]{Software and its engineering~Domain specific languages}
}

\keywords{workflow, asynchronous exception, checkpoint, monad, embedded
  domain specific language}


\RelatedArticle{H. Inoue, T. Aotani and A. Igarashi, ``ContextWorkflow: A Monadic DSL for Compensable and Interruptible Executions'', in Proceedings of the 32nd European Conference on Object-Oriented Programming (ECOOP 2018), LIPIcs, Vol.~0, pp.~2:1--2:30, 2018.\newline \url{http://dx.doi.org/10.4230/LIPIcs.ECOOP.2018.2}}

\Volume{4}
\Issue{3}
\Article{1}
\RelatedConference{32nd European Conference on Object-Oriented Programming (ECOOP 2018), July 19--21, 2018, Amsterdam, Netherlands}


\begin{document}

\maketitle

\begin{abstract}
  This artifact provides the Scala, Haskell, and Purescript
  implementations of ContextWorkflow, an embedded domain-specific
  language for interruptible and compensable executions, and
  demonstrates the maze search example described in the companion
  paper. The Haskell and Purescript implementations provide the core
  language constructs including \texttt{checkpoint} for partial aborts
  and \texttt{sub} for sub-workflows and show that ContextWorkflow can
  be embedded in eager and lazy languages as described in the
  companion paper. The Scala implementation does not only provide
  user-friendly syntax of ContextWorkflow but also gives the maze
  search example as an interactive GUI application.
 \end{abstract}

 \begin{scope}
   The artifact is designed to support the feasibility of (1) our
   monadic embedding of ContextWorkflow to eager and lazy languages
   and (2) the maze search example given in the companion paper.
\end{scope}

\begin{content}
The artifact package includes:
\begin{itemize}
\item A virtual machine image \texttt{ContextWorkflow.ova} that
  contains under \texttt{/home/cworkflow/contextworkflow}
  \begin{itemize}
  \item Scala, Haskell, and Purescript implementations of
    ContextWorkflow with tiny test programs
  \item The maze search robot simulator with GUI
  \end{itemize}
\item A zip archive \texttt{cw-sources.zip} that contains the source
  code of the three implementations
\item A document \texttt{artifact.pdf} that provides instructions for
  running the implementations
\item The md5 sums of the three files
\end{itemize}
\end{content}

\begin{getting}
\end{getting}

\begin{platforms}
  The artifact disk image is known to work on any platform running
  Oracle VirtualBox version 5 (\url{https://www.virtualbox.org/}) with
  5 GiB of free disk space and 2 GiB of free RAM.
\end{platforms}

\license{The artifact is available under MIT license.}

\mdsum{5cbb66a47d17e765b82d738e5532c951}

\artifactsize{3.6 GiB}

\subparagraph*{Acknowledgements.}

We thank the kind and patient reviewers of the artifact for their helpful comments. 


\end{document}
