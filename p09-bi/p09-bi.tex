\documentclass[a4paper,UKenglish]{darts-v2018}
%This is a template for producing DARTS artifact descriptions. 
%for A4 paper format use option "a4paper", for US-letter use option "letterpaper"
%for british hyphenation rules use option "UKenglish", for american hyphenation rules use option "USenglish"
% for section-numbered lemmas etc., use "numberwithinsect"
 
\usepackage{microtype}%if unwanted, comment out or use option "draft"

\usepackage{xspace}   % Smart spacing

\nolinenumbers

%\graphicspath{{./graphics/}}%helpful if your graphic files are in another directory

%\nolinenumbers to disable line numbers

\bibliographystyle{plainurl}% the recommended bibstyle

% Commands for artifact descriptions
% Written by Camil Demetrescu and Erik Ernst
% April 8, 2014

\newenvironment{scope}{\section{Scope}}{}
\newenvironment{content}{\section{Content}}{}
\newenvironment{getting}{\section{Getting the artifact} The artifact 
endorsed by the Artifact Evaluation Committee is available free of 
charge on the Dagstuhl Research Online Publication Server (DROPS).}{}
\newenvironment{platforms}{\section{Tested platforms}}{}
\newcommand{\license}[1]{{\section{License}#1}}
\newcommand{\mdsum}[1]{{\section{MD5 sum of the artifact}#1}}
\newcommand{\artifactsize}[1]{{\section{Size of the artifact}#1}}

\newcommand{\orcid}[1]{\url{http://orcid.org/#1}}
\newcommand{\email}[1]{\href{mailto:#1}{\texttt{#1}}}

\newcommand\name{\textsf{SEDEL}\xspace}


% Author macros::begin %%%%%%%%%%%%%%%%%%%%%%%%%%%%%%%%%%%%%%%%%%%%%%%%
\title{Typed First-Class Traits (Artifact)}
% \titlerunning{A Sample DARTS Research Description} %optional, in case that the title is too long; the running title should fit into the top page column

% ARTIFACT: Authors may not be exactly the same as the related scholarly paper, e.g., you may want to include authors who contributed to the preparation of the artifact, but not to the companion paper

\author{Xuan Bi$^1$}{The University of Hong Kong, Hong Kong, China}{xbi@cs.hku.hk}{}{}%mandatory, please use full name; only 1 author per \author macro; first two parameters are mandatory, other parameters can be empty.

\author{Bruno C. d. S. Oliveira}{The University of Hong Kong, Hong Kong, China}{bruno@cs.hku.hk}{}{Funded by Hong Kong Research Grant Council projects number 17210617 and 17258816}

\authorrunning{X.\,Bi, B.\,C.\,d.\,S.\,Oliveira and T.\,Schrijvers} %mandatory. First: Use abbreviated first/middle names. Second (only in severe cases): Use first author plus 'et. al.'

\Copyright{Xuan Bi and Bruno C. d. S. Oliveira}%mandatory, please use full first names. LIPIcs license is "CC-BY";  http://creativecommons.org/licenses/by/3.0/


\subjclass{Software and its engineering $\rightarrow$ Object oriented languages}% mandatory: Please choose ACM 2012 classifications from https://www.acm.org/publications/class-2012 or https://dl.acm.org/ccs/ccs_flat.cfm . E.g., cite as "General and reference $\rightarrow$ General literature" or \ccsdesc[100]{General and reference~General literature}.

\keywords{traits; extensible designs}%mandatory
% Author macros::end %%%%%%%%%%%%%%%%%%%%%%%%%%%%%%%%%%%%%%%%%%%%%%%%%

% Please provide information to the related scholarly information
\RelatedArticle{Xuan Bi and Bruno C. d. S. Oliveira, ``Typed First-Class Traits'', in Proceedings of the 32nd European Conference on Object-Oriented Programming (ECOOP 2018), LIPIcs, Vol.~0, pp.~0:1--0:2, 2018.\newline \url{http://dx.doi.org/10.4230/LIPIcs.xxx.xxx.xxx}}

%Editor-only macros:: begin (do not touch as author)%%%%%%%%%%%%%%%%%%%%%%%%%%%%%%%%%%
\Volume{4}
\Issue{3}
\Article{9}
\RelatedConference{32nd European Conference on Object-Oriented Programming (ECOOP 2018), July 19--21, 2018, Amsterdam, Netherlands}
% Editor-only macros::end %%%%%%%%%%%%%%%%%%%%%%%%%%%%%%%%%%%%%%%%%%%%%%%

\begin{document}

\maketitle

\begin{abstract}
  This artifact contains the prototype Haskell implementation of \name, with
  support for first-class traits, as described in the companion paper. This artifact
  also contains the source code of the case study on ``Anatomy of Programming
  Languages'', illustrating how effective \name is in terms of modularizing
  programming language features. For comparison, it also includes a vanilla Haskell implementation
  of the case study without any code reuse.
 \end{abstract}

% ARTIFACT: please stick to the structure of 7 sections provided below

% ARTIFACT: section on the scope of the artifact (what claims of the paper are intended to be backed by this artifact?)
\begin{scope}
  This artifact contains the prototype Haskell implementation of \name (a REPL
  and a type-checker), with support for first-class traits, as described in the
  companion paper. This artifact also contains the source code of the case study
  on ``Anatomy of Programming Languages'', illustrating how effective \name is
  in terms of modularizing programming language features. For comparison, it
  also includes a vanilla Haskell implementation of the case study without any
  code reuse.There are also several \name examples to get the readers familiar
  with its syntax.
\end{scope}

% ARTIFACT: section on the contents of the artifact (code, data, etc.)
\begin{content}
The artifact package includes:
\begin{itemize}
\item \texttt{impl} directory: Haskell project containing the source code of \name;
\item \texttt{examples} directory: the source code of case study, as well as some \name examples;
\item \texttt{haskell} directory: Haskell implementation of the case study;
\item \texttt{README.pdf} : Instructions of how to build the artifact.
\end{itemize}
\end{content}

% ARTIFACT: section containing links to sites holding the
% latest version of the code/data, if any
\begin{getting}
% leave empty if the artifact is only available on the DROPS server.
% otherwise, provide links to the latest version of the artifact (e.g., on github)
In addition, the artifact is also available at:
\url{https://github.com/bixuanzju/first-class-trait}.
\end{getting}

% ARTIFACT: section specifying the platforms on which the artifact is known to
% work, including requirements beyond the operating system such as large
% amounts of memory or many processor cores
\begin{platforms}
  The artifact is known to work on any platform running GHC (version 8.2.2 or later).
\end{platforms}

% ARTIFACT: section specifying the license under which the artifact is
% made available
\license{BSD}

% ARTIFACT: section specifying the md5 sum of the artifact master file
% uploaded to the Dagstuhl Research Online Publication Server, enabling 
% downloaders to check that the file is the expected version and suffered 
% no damage during download.
\mdsum{ADDFF0333B277CBD118DAFD23D578FC1}

% ARTIFACT: section specifying the size of the artifact master file uploaded
% to the Dagstuhl Research Online Publication Server
\artifactsize{485 KB}

\subparagraph*{Acknowledgements.}

We would like to thank the anonymous reviewers for their helpful comments
and suggestions.

% ARTIFACT: optional appendix
% \appendix
% \section{Morbi eros magna}

% Morbi eros magna, vestibulum non posuere non, porta eu quam. Maecenas vitae orci risus, eget imperdiet mauris. Donec massa mauris, pellentesque vel lobortis eu, molestie ac turpis. Sed condimentum convallis dolor, a dignissim est ultrices eu. Donec consectetur volutpat eros, et ornare dui ultricies id. Vivamus eu augue eget dolor euismod ultrices et sit amet nisi. Vivamus malesuada leo ac leo ullamcorper tempor. Donec justo mi, tempor vitae aliquet non, faucibus eu lacus. Donec dictum gravida neque, non porta turpis imperdiet eget. Curabitur quis euismod ligula \cite{DBLP:books/mk/GrayR93,DBLP:conf/focs/FOCS16,DBLP:conf/focs/HopcroftPV75,DBLP:journals/cacm/Dijkstra68a,DBLP:journals/cacm/Knuth74}.


% ARTIFACT: include here any additional references, if needed...

%%
%% Bibliography
%%

%% Either use bibtex (recommended), 

% \bibliography{darts-v2018-sample-article}

%% .. or use the thebibliography environment explicitely



\end{document}
